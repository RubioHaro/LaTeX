\section{Introducción}
    Latex es un sistema de preparación de documentos.
    \subsection{Uso}
    Por sus características y posibilidades, es usado de forma especialmente intensa en la generación de artículos y libros científicos que incluyen, entre otros elementos, expresiones matemáticas.
    \subsection{Nombre y pronunciación}
    Lamport ha dicho: Uno de los problemas más grandes de LaTeX es decidir como pronunciarlo. Esto es una de las pocas cosas que no voy a decir acerca de LaTeX, dado que la pronunciación es determinada por el uso, no por las reglas. TeX es usualmente pronunciado teck, haciendo de lah-teck y lay-teck las elecciones más lógicas; pero el lenguaje no siempre es lógico, por ello Lay-tecks también es posible
    \section{Instalación}
    \subsection{ArchLinux}
    Para instalar LaTeX en ArchLinux podemos hacer uso de los paquetes AUR.
    LaTeX se compone de dos paquetes: el core y el lib.